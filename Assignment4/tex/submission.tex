\documentclass{article}

\usepackage[margin=1in]{geometry}
\usepackage{amsmath}
\usepackage{amssymb}
\usepackage{graphicx}

\graphicspath{{./../figures/}}

\title{Assignment 4}
\author{Vignesh M Pai (20211132)}
\date{}

\begin{document}

\maketitle

\begin{center}
    \includegraphics*[scale=0.8]{1a.png}
\end{center}

\begin{center}
    \includegraphics*[scale=0.8]{1b.png}
\end{center}

\section*{1}

$y_A - y_E$ = \input{../data/q1.dat}

\section*{2}

$y_A - y_{ME}$ = \input{../data/q2.dat}

\section*{3}

$y_A - y_{IE}$ = \input{../data/q3.dat}

\section*{4}

$y_A - y_{RK}$ = \input{../data/q4.dat}

\begin{center}
    \includegraphics*[scale=0.8]{7x.png}
\end{center}

\begin{center}
    \includegraphics*[scale=0.8]{7v.png}
\end{center}

\begin{center}
    \includegraphics*[scale=0.8]{7E.png}
\end{center}

The equation of motion for a pendulum is $ml^2 \ddot{\theta} = -mg l \sin \theta$ and the initial angular velocity for which the particle goes around the circle is
\begin{align*}
    \frac{1}{2}ml^2 \omega^2 = 2 m g l \implies \omega = 2 \sqrt{\frac{g}{l}}
\end{align*}

\section*{5}

With $v_0 = 1.9$, $x(50)$ = \input{../data/q5.dat}

\section*{6}

With $v_0 = 1.999$, $x(50)$ = \input{../data/q6.dat}

\section*{7}

With $v_0 = 2.01$, $x(50)$ = \input{../data/q7.dat}

\section*{8}

$y_1(40)$ = \input{../data/q8.dat}

\section*{9}

\begin{center}
    \includegraphics*[scale=0.8]{9.png}
\end{center}

$y(0.8)$ = \input{../data/q9.dat}

\end{document}