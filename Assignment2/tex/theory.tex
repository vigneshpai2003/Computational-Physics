\documentclass{article}

\usepackage[margin=1in]{geometry}
\usepackage{amsmath}

\title{Random Numbers and Monte Carlo Integration}
\author{Vignesh M Pai (20211132)}
\date{}

\begin{document}

\maketitle

\section{Transforming One Dimensional Probability Distributions}

We start with a random number generator $X$ that uniformly gives a random number in the
range $[0, 1]$. The problem is to transform this number by a bijective function $f: [0, 1] \to [a, b]$ such that the resulting
random function $Y=f(X)$ has probability distribution function $\rho$. In other words, we need to satisfy the equation
\begin{align}\label{eq1}
    p(Y \in [y, y + dy]) = \rho(y) dy
\end{align}

We know that
\begin{align}
    p(Y \in [f(x), f(x) + f'(x) dx]) = p(X \in [x, x + dx]) = dx
\end{align}
substituting $y = f(x)$ and $dy = f'(x) dx$, we get
\begin{align}
    p(Y \in [y, y + dy]) = \frac{dy}{f'(x)} = \frac{dy}{f'(f^{-1}(y))}
\end{align}
Let the inverse of $f$ be $g$, then
\begin{align}
    f'(f^{-1}(y)) = f'(g(y)) = \frac{1}{g'(y)}
\end{align}

Substituting back to (\ref{eq1}), we get that
\begin{gather*}
    g'(y) = \rho(y) \\
    f^{-1}(y) = \int_a^y \rho(y) dy
\end{gather*}
Hence, to find $f$, $\rho$ needs to be integrable and the integral needs to be invertible.
Let's calculate $f$ for some common values of $\rho$.

\begin{enumerate}
    \item
    \begin{align*}
        \rho(x) &= e^{-\lambda x},\quad x \geq - \frac{\ln{\lambda}}{\lambda} \\
        g(y) &= 1 - \frac{e^{-\lambda y}}{\lambda} \\
        f(x) &= - \frac{\ln(\lambda(1- x))}{\lambda}     
    \end{align*}
    \item
    \begin{align*}
        \rho(x) &= \lambda e^{-\lambda x},\quad x \geq 0 \\
        g(y) &= 1 - e^{-\lambda y} \\
        f(x) &= - \frac{\ln(1- x)}{\lambda}  
    \end{align*}
\end{enumerate}

\section{Transforming the Gaussian Distribution}

Suppose we wish to calculate $f$ for $\rho$ given by
\begin{align*}
    \rho(x) = \frac{1}{\sqrt{2\pi}} \exp\left(-\frac{x^2}{2}\right)
\end{align*}
Clearly, this is not integrable, hence calculating $f$ is not so simple.
However, we know that $\rho'(x, y) = \rho(x)\rho(y)$ is integrable once transformed to polar coordinates.
\begin{align*}
    \rho'(x, y) dx dy &= \rho'(r \cos(\theta), r \sin(\theta)) r dr d\theta \\
    &= \frac{1}{2 \pi} \exp\left(-\frac{r^2}{2}\right) r dr d\theta \\
    &= - d \left(\exp\left(-\frac{r^2}{2}\right) \right) d\left(\frac{\theta}{2\pi}\right)
\end{align*}
If we choose the change of variables
\begin{align*}
    U &= \exp\left(-\frac{r^2}{2}\right) \\
    V &= \frac{\theta}{2\pi}
\end{align*}
then note that $U, V$ lie in the range $[0, 1]$. If $U, V$ set to be random variables,
and we write $x, y$ as
\begin{align*}
    x &= r \cos(\theta) = \sqrt{-2 \ln U} \cos(2 \pi V) \\
    y &= r \sin(\theta) = \sqrt{-2 \ln U} \sin(2 \pi V)
\end{align*}
then $x, y$ will each be random variables with the distribution $\rho$.
The proof of this fact essentially lies in the change of variables we performed to derive $U, V$.

It turns out trigonometric functions are expensive to compute. Therefore, in practice, we can use the Marsaglia polar method.
Let $u, v$ be random variables in $[-1, 1]$ (they represent $\sqrt{U} \cos{\theta}$ and $\sqrt{U} \sin {\theta}$), let $s = u^2 + v^2$ and continue until $s \leq 1$.
We now write $x, y$ as
\begin{align*}
    x &= \sqrt{-2\ln s} \frac{u}{\sqrt{s}} = \sqrt{\frac{-2\ln s}{s}} u \\
    y &= \sqrt{-2\ln s} \frac{v}{\sqrt{s}} = \sqrt{\frac{-2\ln s}{s}} v
\end{align*}
It can be verified that $x, y$ have distribution $\rho$.

Note that if a random variable $X$ has distribution $\rho$, then $\sigma X + \mu$ has Gaussian distribution with mean $\mu$ and variance $\sigma^2$.

\end{document}