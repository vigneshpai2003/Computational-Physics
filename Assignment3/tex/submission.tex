\documentclass{article}

\usepackage[margin=1in]{geometry}
\usepackage{amsmath}
\usepackage{amssymb}
\usepackage{graphicx}

\graphicspath{{./../figures/}}

\title{Assignment 3}
\author{Vignesh M Pai (20211132)}
\date{}

\begin{document}

\maketitle

\section*{1}

The total magnetic moment will be $-L^3$ or $-8000$.

\section*{2}

The number of interactions is $6$ per spin and thus $6L^3 / 2 = 3 L^3$.
Since all interactions will contribute to $-J$ interaction energy, the total energy will be $-3 J L^3 = -3000$.

\section*{3}

\begin{center}
    \includegraphics*[scale=0.8]{3.png}
\end{center}

\section*{4}

\begin{center}
    \includegraphics*[scale=0.8]{4.png}
\end{center}

\section*{5}

\begin{center}
    \includegraphics*[scale=0.8]{5.png}
\end{center}

\section*{6}

\begin{center}
    \includegraphics*[scale=0.8]{6M.png}
\end{center}

\begin{center}
    \includegraphics*[scale=0.8]{6E.png}
\end{center}

\section*{Figures}

\begin{center}
    \includegraphics*[scale=0.8]{fM.png}
\end{center}

\begin{center}
    \includegraphics*[scale=0.8]{fE.png}
\end{center}

\begin{center}
    \includegraphics*[scale=0.8]{fchi.png}
\end{center}

\begin{center}
    \includegraphics*[scale=0.8]{fCv.png}
\end{center}

\section*{7}

The value of $\chi$ at $T=4.5$ for $L=7, 8, 9$ respectively are \input{../data/7.dat}.

\section*{8}

The peak value of $C_V$ for $L = 8$ is \input{../data/8.dat}.

\section*{9}

The peak value of $C_V$ for $L = 9$ is \input{../data/9.dat}.

\section*{10}

The value of magnetization per spin for $L=7$ at $T=3.8$ is \input{../data/10.dat}.

\section*{11}

\begin{center}
    \includegraphics*[scale=0.8]{BC.png}
\end{center}

\begin{center}
    \includegraphics*[scale=0.8]{BCz.png}
\end{center}

\section*{12}

By detailed balance, we must have
\begin{align*}
    n_A r_{A \to B} = n_B r_{B \to A}
\end{align*}
Hence, 20 particles must be jumping from $E_{10}$ to $E_5$.

\end{document}